\chapter{Conclusions}
\label{cha:coclusion}

This work provided an overview of soft robot evolution using Evolution Gym benchmark.\\
Soft robotics is a promising brand new robotic research field.
The main feature of soft robots, which is their deformability and adaptability to complex and unpredictable environments, makes them an interesting subject. 
Not only are they bio-inspired, but also they will have many future bio-related applications, and their malleability and adaptability will lead to an enhanced human-machine interaction.

A focus on the controller optimization showed that it has a key role in soft robot evolution, since it allows all the individuals to perform their best, even the ones with a disadvantaged morphology.

A comparison of two design algorithms, GA and MAP-Elites, showed that focusing on the mere fitness goal does lead to greater fitness values, but it generates similar bodies, precluding from exploring new potentially well-fitting morphologies. On the contrary, focusing on diversity allows the evaluation of very different robot designs, and it leads to only slightly lower performances.

One of the main reasons why soft robots are considered a promising field is their flexibility and adaptability, and therefore their potential application in different environments.
This work revealed that even though the best results are obtained when both the controller and the body are optimized in the same task, it's also possible to get good results in a new environment simply by re-training the controller.\\
It was also shown that in some environments it's also possible to use the already optimized controller; this avoids the optimization costs and, in some cases, turned out to be even more effective.
However, these results depend on the reward definition, and further experiments might take to more accurate results.

The robot bodies evolved autonomously often resembled existing natural creatures, even with no prior knowledge: for instance, they grew legs and set space to carry an object, according to the goal of the task they have been trained on. This demonstrates once again the strong connection that persists between artificial intelligence and biology.

Nevertheless, this work only focused on three easy tasks. The benchmark used proposes a great variety of tasks of different difficulty, therefore it would be interesting to extend the study of soft robot evolution to more complex environments.\\
The design optimization was limited to two evolutionary algorithms, but many other exist and might lead to interesting results.\\
What is more, only two body-related features were considered, the actuation and the emptiness, therefore a deeper study might analyze other features, like the energy consumption, relevant for future applications.

This work only provided an overview of soft robot evolution, but many encouraging aspects emerged yet. This is a promising field that deserves to be explored, and further studies will surely take steps towards an enhanced human-machine interaction.

\newpage
