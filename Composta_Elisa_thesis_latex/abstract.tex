\chapter*{Abstract} % senza numerazione
\label{abstract}

\addcontentsline{toc}{chapter}{Abstract} % da aggiungere comunque all'indice

An important and promising robotic subject that is gaining ground is Soft Robotics, a sub-field that examines robots with a soft and deformable body.\\
What makes these individuals so powerful, is that their malleable morphology allows them to face complex and dynamic environments and gives them the chance to perform well even on the hardest tasks.\\
This is a promising area, since it will enhance human-machine interaction, having many future bio-related applications, like prostheses, artificial organs, and support for gait rehabilitation.
However, it sets many challenges due to the complexity of finding individuals that are both soft and robust, able to adapt to unpredictable environments, therefore it requires deeper studies.

Robots, just like living creatures, should be a good match of body and brain.
It's easy to imagine that some bodies are intrinsically better than others: an individual with two flexible legs walks better than a rigid squared box.
Another simple and intuitive idea is that the brain certainly plays an equally important role: what is a body without a brain?
Even the best bodies can have poor results, having no control of their actions.\\
Given this premises, it is evident that body and brain must be both optimized, in order make individuals perform their best.
However, while many researches focus on the controller optimization, less attention is placed on finding the best bodies, mainly because the co-optimization of body and brain is still a challenging problem.

This work addresses the co-optimization problem of soft robots using Reinforcement Learning to optimize the controller and applying Evolutionary Algorithms to design morphologies.\\
It presents the importance of co-designing both the body and the controller and it makes a comparison between two existing evolutionary algorithms, MAP-Elites and the Genetic Algorithm.\\
A focus on the controller optimization highlights its role in improving the performances, since it allows also disadvantaged body to perform their best.

An analysis on the adaptability of individuals was made by evaluating them in environments they have not been trained for, optimizing the controller in the new task.

Even though this work only provides an overview of soft robot evolution, what emerges is that this is a promising field, and further studies might lead to interesting results, with many possible future applications.



